\chapter{Conclusion}

The purpose of this report is to address the issue of oscillation in the temperature of the oil in an ORC (organic rankine cycle) system. This oscillation is caused by the heater in the oil cycle overshooting or undershooting the desired temperature. To solve this problem, a dynamic model was developed using Matlab/Simulink, which incorporates thermodynamic and heat transfer equations to simulate the various components of the oil cycle. This model was then validated by comparing it to experimental data, which showed that the model accurately represents the real system. To address the issue of oscillation, a solution was proposed in which a bypass pipeline is created between the inlet and exit of the oil cycle, allowing for reverse flow. This solution was simulated using the validated model, and the results showed that the oscillations could be significantly reduced by adjusting the flow through the bypass pipeline. However, since the resistor can not endure too high temperatures, there is a limit to the set temperature, which bounds the mass flow rate that can flow through the bypass line. Therefore, the oscillations cannot be reduced a lot when the set temperature is high. To test the proposed solution in a real system, an experimental concept setup was designed using a pump, a check valve, a 3-way valve, and an actuator. However, the actual experiment only utilized a 3-way valve and an actuator to demonstrate that the flow direction of the fluid could be controlled using these two components.